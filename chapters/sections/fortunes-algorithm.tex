\section{Fortune's algorithm}
In this section we will present an algorithm which computes $\Vor(P)$ in $\mathcal{O}(n \log n)$ time. 

%%%%

\newpage
\begin{prop} \label{prop:highschool1}
Let $f(x) = a x^2 + b x + c$ be a polynomial with discriminant $D > 0$ with roots $r_1 < r_2$. Then $r = \tfrac{1}{2}(r_1 + r_2)$ is the only solution to $\displaystyle\frac{df}{dx}(r) = 0$ and the expressions $\displaystyle\frac{df}{dx}(r_1)$ and $\displaystyle\frac{df}{dx}(r_2)$ are non-zero and have opposite signs.
\end{prop}
\begin{proof}
It is well-known that we may factor $f$ as follows:
\[
    f = a (x - r_1) (x - r_2) = a x^2 - a(r_1 + r_2) x + a r_1 r_2.
\]
Since two polynomials are equal if and only if their coefficients are equal we get $b = - a (r_1 + r_2)$, which gives us
\[
    \frac{df}{dx} (r) = 2 a r + b = 2 a \para{\frac{r_1 + r_2}{2}} - a (r_1 + r_2) = 0.
\]
This is the only solution since $\displaystyle\frac{df}{dx}$ is a first degree polynomial. Now note that $\displaystyle\frac{d^2 f}{d x^2}(x) = 2a \ne 0$ and $r_1 < r < r_2$ which gives us that
\[
    \text{sgn} \para{\frac{df}{dx}(r_1)} = -\text{sgn} \para{\frac{df}{dx}(r_2)} \ne 0.
\]
\end{proof}
\todo{Put a cute geometrical proof of Proposition \ref{prop:highschool1} in a margin figure: Draw a parabola which intersects the $x$-axis twice at $r_1 < r_2$, and note that since the polynomial is symmetrical about its vertex, the expression $r = (r_1 + r_2)/2$ is immediate.}

Proposition \ref{prop:highschool1} will be used when we want to find out which arc on the beach line lies above a new point discovered through a site event. When intersecting two of the paraboals of the beach line, we will find two intersection points, because of our assumptions. Proposition \ref{prop:highschool1} then gives us that at these intersection points $r_1, r_2$ we have that
\[
    \begin{cases}
        \displaystyle\frac{d(\beta_i - \beta_j)}{dx}(r_k) \ne 0 \text{ for } k = 1, 2 & \text{ }\vspace{0.25cm} \\ \text{sgn} \para{\displaystyle\frac{d(\beta_i - \beta_j)}{dx}(r_1)} = -\text{sgn} \para{\displaystyle\frac{d(\beta_i - \beta_j)}{dx}(r_2)} & \text{ }
    \end{cases}
\]
We then want to locate a specific breakpoint between two arcs, and the above will help us to do this.

To intersect two parabolas $\beta_i$ and $\beta_j$ we write
\[
    (\beta_i - \beta_j)(x) = a x^2 + b x + c,
\]
where (for $p = p_i$, $q = p_j$, $h_p = p_y - \ell_y$ and $h_q = q_y - \ell_y$)
\begin{align*}
    a &= \frac{1}{2} \para{\frac{1}{h_p} - \frac{1}{h_q}}, \\
    b &= \frac{q_x}{h_q} - \frac{p_x}{h_p}, \\
    c &= \frac{q_y(p_x^2 + p_y^2) - p_y (q_x^2 + q_y^2) + \ell_y (q_x^2 + q_y^2 - p_x^2 - p_y^2) + \ell_y^2 (p_y - q_y)}{2 h_p h_q}.
\end{align*}
The square root of the discriminant is then
\[
    d = \sqrt{b^2 - 4 ac} = \sqrt{\frac{(p_x - q_x)^2 + (p_y - q_y)^2}{h_p h_q}}.
\]
The $x$-valuees of the intersection points are then given by the well-known formulas
\[
    r_1 = \frac{-b - d}{2 a}, \quad
    r_2 = \frac{-b + d}{2 a},
\]
which gives us the intersection points $q_1 = (r_1, \beta_i(r_1))$ and $q_2 = (r_2, \beta_i(r_2))$. Now, we want to find the breakpoint which at which an arc of $\beta_i$ exits the beach line, and an arc of $\beta_j$ enters the beach line, Proposition \ref{prop:highschool1} gives us a way of picking which one of $q_1$ and $q_2$ is the breakpoint that we need. For $\beta_i$ to exit and $\beta_j$ to enter, we need to pick $k$ such that
\[
    \frac{d \beta_i}{dx}(r_k) > \frac{d \beta_j}{dx}(r_k).
\]
Proposition \ref{prop:highschool1} garantuees that either
\begin{align*}
    \frac{d \beta_i}{dx}(r_1) > \frac{d \beta_j}{dx}(r_1) &\text{ and } \frac{d \beta_i}{dx}(r_2) < \frac{d \beta_j}{dx}(r_2) \\
    &\text{or} \\
    \frac{d \beta_i}{dx}(r_1) < \frac{d \beta_j}{dx}(r_1) &\text{ and } \frac{d \beta_i}{dx}(r_2) > \frac{d \beta_j}{dx}(r_2),
\end{align*}
so it is possible to make the right choice. Now, note that
\[
    \frac{d \beta_i}{dx}(r_k) > \frac{d \beta_j}{dx}(r_k)
\]
if and only if
\[
    (r_k - p_x) (q_y - \ell_y) > (r_k - q_x) (p_y - \ell_y).
\]

\newpage
\subsubsection*{Intersecting bisectors and finding the circle through 3 points}
As a part of the algorithm, we need to find the circle $C$ through 3 points $p_1, p_2, p_3$. It turns out if we intersect $\bi(p_1, p_2)$ and $\bi(p_2, p_3)$ we find the center of $C$, and then to find the radius we just need to find the distance from the center to one of the points. This is because if $x \in \bi(p_1, p_2) \cap \bi(p_2, p_3)$ then
\[
    \dist(x, p_1) = \dist(x, p_2) = \dist(x, p_3),
\]
so $x$ is exactly the center of a circle through $p_1, p_2, p_3$.

To intersect the bisectors, we form the midpoints
\[
    m_1 = \frac{1}{2}(p + q) \quad \text{and} \quad m_2 = \frac{1}{2}(q + r)
\]
and then we let $d_1$ and $d_2$ denote $q - p$ and $r - q$ rotated 90 degrees counterclockwise. Then $s \mapsto m_1 + s d_1$ and $t \mapsto m_2 + t d_2$ parametrize $\bi(p_1, p_2)$ and $\bi(p_2, p_3)$, respectively. To intersect $\bi(p_1, p_2)$ and $\bi(p_2, p_3)$ we then need to find $s, t \in \R$ such that
\[
    m_1 + s d_1 = m_2 + t d_2.
\]
This can be rewritten into the matrix equation
\[
    \begin{pmatrix}
        \mid & \mid \\
        d_1 & d_2 \\
        \mid & \mid
    \end{pmatrix} \begin{pmatrix}
        s \\
        -t
    \end{pmatrix}
    =
    m_2 - m_1,
\]
and if $d_1$ and $d_2$ are linearly independent then it has the unique solution
\[
    \begin{pmatrix}
        s \\
        -t
    \end{pmatrix}
    =
    \begin{pmatrix}
        \mid & \mid \\
        d_1 & d_2 \\
        \mid & \mid
    \end{pmatrix}^{-1} (m_2 - m_1).
\]
This is implemented in the code as the function \texttt{intersectBisectors} and used in the fucntion \texttt{circleThroughThreePoints}.

%Let $(x_1, y_1), (x_2, y_2), (x_3, y_3) \in \R^2$ denote the points. To find the circle, we need to find its center $(a,b) \in \R^2$ and its radius $r > 0$. These will satisfy
%\begin{align}
%    (x_1 - a)^2 + (y_1 - b)^2 - r^2 &= 0, \label{align:circleeq1} \\
%    (x_2 - a)^2 + (y_2 - b)^2 - r^2 &= 0, \label{align:circleeq2} \\
%    (x_3 - a)^2 + (y_3 - b)^2 - r^2 &= 0. \label{align:circleeq3}
%\end{align}
%By subtracting (\ref{align:circleeq1}) from (\ref{align:circleeq2}) and subtracting (\ref{align:circleeq1}) from (\ref{align:circleeq3}) and rearranging we obtain
%\begin{align}
%    2 a (x_1 - x_2) + 2 b (y_1 - y_2) &= x_1^2 - x_2^2 + y_1^2 - y_2^2, \\
%    2 a (x_1 - x_3) + 2 b (y_1 - y_3) &= x_1^2 - x_3^2 + y_1^2 - y_3^2.
%\end{align}
%This can be rewritten in matrix form as
%\[
%    A
%    \begin{pmatrix}
%        a \\ b
%    \end{pmatrix}
%    =
%    \begin{pmatrix}
%        x_1^2 - x_2^2 + y_1^2 - y_2^2 \\
%        x_1^2 - x_3^2 + y_1^2 - y_3^2
%    \end{pmatrix}
%    \quad
%    \text{ where }
%    \quad
%    A = 2 \begin{pmatrix}
%        x_1 - x_2 & y_1 - y_2 \\
%        x_1 - x_3 & y_1 - y_3
%    \end{pmatrix}.
%\]
%It is well-known from linear algebra that $\det(A) = \det(A^T)$, and that the above system has a unique solution if and only if $\det(A) \ne 0$. We have that $\det(A^T) = 0$ if and only if the points are collinear. Now, assuming the %points are not collinear, the unique solution for $(a, b)$ is then given by
%\begin{align*}
%    a &= \frac{2 (x_1^2 - x_2^2 + y_1^2 - y_2^2) (y_1 - y_3) - 2 (x_1^2 - x_3^2 + y_1^2 - y_3^2) (y_1 - y_2)}{\det(A)}, \\
%    b &= \frac{2 (x_1^2 - x_3^2 + y_1^2 - y_3^2) (x_1 - x_2) - 2 (x_1^2 - x_2^2 + y_1^2 - y_2^2) (x_1 - x_3)}{\det(A)},
%\end{align*}
%where
%\[
%    \det(A) = 4 (-x_2 y_1 + x_3 y_1 + x_1 y_2 - x_3 y_2 - x_1 y_3 + x_2 y_3)
%\]
%Once the center $(a, b)$ of the circle $C$ has been found, we can find its radius $r$ by computing the distance from one of the three points to the center, for example
%\[
%    r = \norm{(x_1, y_1) - (a, b)}.
%\]%