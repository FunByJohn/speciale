\section{Fortune's algorithm}
In this section we will present an algorithm which computes $\Vor(P)$ in $\mathcal{O}(n \log n)$ time. This is actually optimal, as we can use a Voronoi diagram for sorting:
\begin{thm} \label{thm:voronoicansort}
We can't do better than $\mathcal{O}(n \log n)$.
\end{thm}
\begin{proof}
Let $A = \curly{a_1, a_2, \ldots, a_n} \subset \R$. Now assume we have used an algorithm to compute a Voronoi diagram of the points
\[
    P = \curly{(a_1, 0), (a_2, 0), \ldots, (a_n, 0)}.
\]
We obtain a diagram which looks similar to this:
\[
    \includegraphics[scale=0.2]{voronoi-sorting}
\]
We assume without loss of generality that the algorithm outputs a DCEL $\Delta$ of $\Vor(P)$. Assume that the $\textsf{edge}$ pointer of every face of $\Delta$ points to the edge to the right of the face, and that the $\textsf{face}$ pointer of every edge of $\Delta$ points to the face to the right. Let $F_i$ be the face in $\Delta$ which contains the point $(0, a_i)$. Let $\ell \in \N$ such that $a_{\ell} < a_i$ for all $i \ne \ell$. Let $b_1 = a_{\ell}$ and if $b_i = a_{j}$ and $i < n$ then define $b_{i+1} = a_k$ where $k$ comes from $F_{j}\textsf{.edge.face} = F_k$. Then $(b_1, b_2, \ldots, b_n)$ is the elements of $A$ in sorted order. This means that we can use the Voronoi diagram to sort, which proves the theorem.
\end{proof}

\todo{The statement of the above theorem is temporary. I originally phrased it like so: ``The optimal worst-case running time for computing $\Vor(P)$ is $\Omega(n \log n)$.'' What is the proper terminology here?} \\

\newpage
Consider a point $p = (p_x, p_y) \in \R^2$ and a sweep line $y = \ell_y$ with $p_y > \ell_y$. We define the distance between $p$ and $\ell$ as
\[
    \dist(p, \ell) = p_y - \ell_y.
\]
We can split $\R \times (\ell_y, \infty)$ into two components $R_p$ and $R_{\ell}$ with a common boundary, where
\begin{align*}
    q \in R_p &\implies \dist(q, p) < \dist(q, \ell) \\ 
    q \in R_\ell &\implies \dist(q, p) > \dist(q, \ell)
\end{align*}
and where their boundary $\partial R_p = \partial R_\ell$ is given as follows: We're interested in computing the $q \in \R \times [\ell_y, \infty)$ which satisfy $\dist(q, p) = \dist(q, \ell)$. Let $q = (x, y)$ with $y \geq \ell_y$ be such a point. Since distances are non-negative, it is equivalent to looking at satisfying $\dist(q, p)^2 = \dist(q, \ell)^2$. We have:
\[
    \dist(q, p)^2 = \dist(q, \ell)^2 \iff (p_x - x)^2 + (p_y - y)^2 = (y - \ell_y)^2.
\]
This can be transformed into the equation
\begin{equation}
    2 (p_y - \ell_y) y = x^2 - 2 p_x x + p_x^2 + p_y^2 - \ell_y^2.
\end{equation}
Since $p_y \ne \ell_y$ by assumption, we obtain the parabola:
\begin{equation} \label{eq:parabola}
    y = \frac{1}{2 (p_y - \ell_y)} (x^2 - 2 p_x x + p_x^2 + p_y^2 - \ell_y^2).
\end{equation}
For each point $p_i \in P = \curly{p_1, p_2, \ldots, p_n}$ we now define
\[
    \beta_i(x) = \begin{cases}
        \displaystyle \frac{1}{2 ((p_i)_y - \ell_y)} (x^2 - 2 (p_i)_x x + (p_i)_x^2 + (p_i)_y^2 - \ell_y^2) & \text{if } (p_i)_y > \ell_y \\
        \infty & \text{otherwise}
    \end{cases}
\]
\begin{defn}[Beach line]
The \emph{beach line for the points $P$ with regards to the sweep line $\ell$} is given by the graph of the function
\[
    \textsc{Beach}(x) = \min\curly{\beta_1(x), \beta_2(x), \ldots, \beta_n(x)}.
\]
\end{defn}

\begin{defn}[Breakpoint]
The intersection of two different $\beta_i$ and $\beta_j$ that lie on the beach line is called a \emph{breakpoint}.
\end{defn}

Now we show that the breakpoints exactly trace out $\VorG(P)$ as the sweep line $\ell$ moves from top to bottom.
\begin{thm}
We have the following:
\begin{enumerate}[{(}i{)}]
\item For every sweep line $\ell$: $y = \ell_y$ each breakpoint lies on $\VorG(P)$.
\item For every point $q$ in $\VorG(P)$ there is a position of the sweep line $\ell$ such that $q$ is a breakpoint.
\end{enumerate}
\end{thm}
\begin{proof}
We prove each statement individually:
\begin{enumerate}[{(}i{):}]
    \item Let $\ell$ be the sweep line, and assume that it has one or more breakpoints. Let $q \in \R^2$ be such a breakpoint. Let $\beta_i, \beta_j$ be two parabolas that meet at $q$, corresponding to the points $p_i, p_j$. Then
    \[
        \dist(q, \ell) = \dist(q, p_i) = \dist(q, p_j).
    \]
    The last equality gives us that $q \not\in \mathcal{V}(p_k)$ for all $k$, hence $q \in \VorG(P)$.
    \item \todo{in progress on paper}
\end{enumerate}
\end{proof}

GeoGebra demo of two points and their beach line
\[
    \boxed{\text{\url{https://www.geogebra.org/calculator/zreutrt6}}}
\]