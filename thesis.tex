\documentclass[oneside]{book}

\usepackage[utf8]{inputenc}
\usepackage{amsthm}
\usepackage{amssymb}
\usepackage{amsmath}
\usepackage{dsfont}
\usepackage{xcolor}
\usepackage{natbib}
\usepackage{enumitem}
\usepackage{verbatim}
\usepackage{lipsum}
\usepackage{tikz}
\usepackage{mathtools}
\usepackage{float}
\usepackage{graphicx}
\usepackage{subfig}
\usepackage{wrapfig}

\usepackage{hyperref}
\hypersetup{
    colorlinks,
    citecolor=black,
    filecolor=black,
    linkcolor=black,
    urlcolor=black
}
    
\graphicspath{{"/Users/Johannes/Documents/TeX/Speciale/report/images/"}}
\captionsetup[subfigure]{labelformat=empty}

% Theorem enviroments, which all use the same counter
\theoremstyle{definition}
\newtheorem{thm}{Theorem}
\newtheorem{defn}[thm]{Definition}
\newtheorem{cor}[thm]{Corollary}
\newtheorem{lem}[thm]{Lemma}
\newtheorem{prop}[thm]{Proposition}
\newtheorem{ex}[thm]{Example}
\newtheorem{rmk}[thm]{Remark}
\newtheorem{calc}[thm]{Calculation}
\numberwithin{thm}{chapter}
%\numberwithin{equation}{section}

\numberwithin{equation}{chapter}
\renewcommand{\theequation}{\thechapter.\arabic{equation}}

% Sets
\newcommand{\R}{\mathbb{R}}
\newcommand{\F}{\mathbb{F}}
\newcommand{\N}{\mathbb{N}}
\newcommand{\Z}{\mathbb{Z}}

% Misc
\newcommand{\curly}[1]{\{#1\}}
\newcommand{\abs}[1]{\left\lvert #1 \right\rvert}
\newcommand{\norm}[1]{\left\lVert #1 \right\rVert}
\newcommand{\ip}[2]{\left\langle #1, #2 \right\rangle}
\newcommand{\para}[1]{\left(#1\right)}
\newcommand{\makeset}[2]{\curly{#1 \mid #2}}

% Todo
\newcommand{\todo}[1]{{\color{red} (\textsc{todo}: #1)}}
\newcommand{\ques}[1]{{\color{red} (\textsc{question}: #1)}}

% CompGeo
\DeclareMathOperator{\dist}{dist}
\DeclareMathOperator{\Vor}{Vor}
\DeclareMathOperator{\bi}{bi}

\begin{document}

\begin{titlepage}
\begin{center}
\Huge{\textsc{Computing Voronoi Diagrams Using Fortune's Algorithm}}

\par
\vspace{0.6cm}

\Large{\textsc{Student: Johannes Jensen, 201505594}}

\par
\vspace{0.33cm}

\Large{\textsc{Supervisor: Gerth Stølting Brodal}}

\par
\vspace{0.33cm}

\large{\textsc{Master Thesis in Mathematics}}
\par
\large{\textsc{Aarhus University}}

\par
\vspace{0.25cm}

\normalsize{\textsc{June 15, 2022}}
\end{center}
\[
    \includegraphics[scale=1.3]{frontpage}
\]

\end{titlepage}

\frontmatter
\chapter{Abstract}

The main goal of this master thesis is to solve the following problem: Given a set of $n$ points $P$ in the plane called sites, draw polygons around the sites such that none of the polygons' interiors overlap, every polygon contains only one site, and such that the distance from every point inside a polygon is closer to the site inside the polygon with respect to the Euclidean distance, than it is to any of the other sites. Such a collection of polygons is called an Euclidean Voronoi diagram for the set of sites $P$.

We will work towards presenting an algorithm which runs in $\mathcal{O}(n \log n)$ time, which is called Fortune's algorithm. This is a sweep line algorithm, which sweeps a hypothetical horizontal line down through the plane, uncovering the structure of the Voronoi diagram along the way.

First we look at some theory, where we prove some local and global properties of Euclidean Voronoi diagrams, and how they interact with the sweep line. Afterwards we describe some data structures that we will use, and during this we will describe and prove some properties about treaps, which are randomized binary search trees. Finally, we put everything together and describe the algorithm in detail.

\tableofcontents

\mainmatter
\chapter{Introduction}

Let $\norm{\cdot} \colon \R^2 \to \R$ be a norm. Then we define the distance function as
\begin{equation}
    \dist(p, q) = \norm{p - q}.
\end{equation}
For $1 \leq p < \infty$ we define the $L^p$ norm by
\begin{equation}
    \norm{(x, y)}_p = {\big(\abs{x}^p + \abs{y}^p\big)}^{1/p},
\end{equation}
and we note that $\norm{ \cdot }_2$ is the well-known Euclidean distance. For $p = 1$, the above reduces to
\begin{equation}
    \norm{(x, y)}_1 = \abs{x} + \abs{y}.
\end{equation}
Letting $p \to \infty$, we also obtain the norm
\begin{equation}
    \norm{(x, y)}_{\infty} = \max\big(\abs{x}, \abs{y}\big).
\end{equation}
\begin{defn}[Voronoi diagram]
Let $P = \curly{p_1, p_2, \ldots, p_n} \subset \R^2$. The cells corresponding to each point are denoted by
\[
    \mathcal{V}(p_i) = \makeset{q \in \R^2}{\dist(q, p_i) < \dist(q, p_j) \text{ for all } i \ne j}.
\]
The Voronoi diagram of $P$, denoted $\Vor(P)$, is the subdivision of $\R^2$ consisting of the union of the cells $\mathcal{V}(p_1), \mathcal{V}(p_2), \ldots, \mathcal{V}(p_n)$.
\end{defn}

The following figure shows how the Voronoi diagram for 9 random points looks like with regards to some different $L^p$ norms:

\begin{figure}[H]
    \centering
    \subfloat[$p = 1$]{
      \includegraphics[scale=0.21]{naive-voronoi-L1}
    }
    \subfloat[$p = 2$]{
      \includegraphics[scale=0.21]{naive-voronoi-L2}
    }
    \hspace{0mm}
    \subfloat[$p = 5$]{
      \includegraphics[scale=0.21]{naive-voronoi-L5}
    }
    \subfloat[$p = \infty$]{
      \includegraphics[scale=0.21]{naive-voronoi-Linfty}
    }
    \caption{$\Vor(P)$ of 9 random points using different $\norm{\,\cdot\,}_p$}
    \label{fig:naive-voronoi}
\end{figure}
The above figures were generated using a very naive algorithm, which for each each pixel determinates which of the 9 points is the closest with regards to the chosen norm. A demo is available in $\boxed{\textsf{demos/pixel-voronoi-naive}}$. \\

Note that some of the cells may be unbounded, for example the bottom left green cell in the above figure. For $p = 1$ and $p = \infty$ the boundaries of the cells $\mathcal{V}(p_i)$ are characterised by lines, rays and segments that can only point in the 8 compass directions. For $p = 2$ the boundaries consist of lines, rays and segments which can point in any direction. Interestingly, for $2 < p < \infty$ it seems that the boundary consists of smooth curves that are not necessarily part of a line. \\

We now want to look at the graph structure of the Voronoi diagram. For $P = \curly{p_1, p_2, \ldots, p_n} \subset \R^2$ the set
\[
    \VorG(P) = \R^2 - \Vor(P)
    = \makeset{q \in \R^2}{\dist(q, p_i) = \dist(q, p_j) \text{ for some } i \ne j}
\]
turns out to be an embedding of a graph, where some of the edges are infinite, here's a visualization:
\begin{figure}[H]
    \centering
    \subfloat[$p = 1$]{
      \includegraphics[scale=0.21]{naive-voronoi-graph-L1}
    }
    \subfloat[$p = 2$]{
      \includegraphics[scale=0.21]{naive-voronoi-graph-L2}
    }
    \hspace{0mm}
    \subfloat[$p = 5$]{
      \includegraphics[scale=0.21]{naive-voronoi-graph-L5}
    }
    \subfloat[$p = \infty$]{
      \includegraphics[scale=0.21]{naive-voronoi-graph-Linfty}
    }
    \caption{$\VorG(P)$ of the 9 random points using different $\norm{\,\cdot\,}_p$.}
\end{figure}
The above figures were generated by first generating the images from Figure \ref{fig:naive-voronoi} and then performing the following algorithm: For each pixel, we look at the surrounding pixels within a small disk about that point, and if it contains exactly 2 different colors, we know that we're looking at an edge, so we color the pixel black, and if we see 3 colors or more, we know that we're at a vertex. If we only see 1 color, then we just color the pixel white.

Note that it's the black vertices and edges which make up the graph, the gray points from $P$ are just there for visualization. Rather than computing $\Vor(P)$, our algorithms will actually compute $\VorG(P)$, and from there be able to compute $\Vor(P)$. \\
\chapter{Properties of Euclidean Voronoi Diagrams}

We follow the presentation in Section 7.1 in \cite{CompGeo}.

% % % % % % % % % % % % % % % % % % % % % % % % % % % % % % % % % % % %
%
% Definition
%
% % % % % % % % % % % % % % % % % % % % % % % % % % % % % % % % % % % % 
In this chapter we will consider Voronoi diagrams for the $L^2$ norm, also known as the Euclidean norm. The norm is given by
\[
    \norm{(x, y)}_2 = \sqrt{x^2 + y^2},
\]
for all $x, y \in \R$. Here is the example diagram with this norm from earlier:
\begin{figure}[H]
    \centering
    \subfloat{
      \includegraphics[scale=0.21]{naive-voronoi-L2}
    }
    \subfloat{
      \includegraphics[scale=0.18]{naive-voronoi-graph-L2}
    }
\end{figure}
We note that the diagram consists of straight lines, rays and line segments. In the following sections we will describe the shape of the diagram in detail.

% % % % % % % % % % % % % % % % % % % % % % % % % % % % % % % % % % % %
%
% Preliminary definitions: Bisector, halfplane, Voronoi cells
%
% % % % % % % % % % % % % % % % % % % % % % % % % % % % % % % % % % % % 

\section{Bisectors, halfplanes and Voronoi cells}
From linear algebra we know that $\norm{v}_2 = \sqrt{\ip{v}{v}}$, where $\ip{\,\cdot\,}{\,\cdot\,}$ is the usual dot product on $\R^2$. Given two points $p, q \in \R^2$ then the \textbf{bisector} of $p$ and $q$ is denoted by $\bi(p, q) \subset \R^2$ and denotes the set of points on a line $\ell$ which passes through the midpoint of $p$ and $q$ and is orthogonal (w.r.t. $\ip{\,\cdot\,}{\,\cdot\,}$) to the vector $p - q$.
\[
    \includegraphics[scale=0.8]{bisector} %\includegraphics[scale=0.22]{temp-fig-2}
\]
A bisector $\bi(p, q)$ splits the plane into two \textbf{half-planes} $H_p$ and $H_q$ such that $p \in H_p$ and $q \in H_q$. We define $h(p, q)$ to be the open half-plane which contains $p$, that is the interior of $H_p$. So we have that
\[
    \R^2 = h(p, q) \cup \bi(p, q) \cup h(q, p).
\]
\begin{prop} \label{prop:hyperplaneinclusionproperty}
$r \in h(p, q)$ if and only if $\dist(r, p) < \dist(r, q)$.
\end{prop}
\begin{proof}
Let $r \in h(p, q)$ and let $s$ be the projection of $r$ onto the line segment $\overline{pq}$.
\[
    \includegraphics[scale=0.8]{halfplane_containment} %\includegraphics[scale=0.25]{temp-fig-1}
\]
The Pythagorean theorem and the fact that $\overline{ps}$ is shorter than $\overline{sq}$ then gives us
\begin{align*}
    \norm{p - r}^2 &= \norm{p - s}^2 + \norm{s - r}^2 \\
    &< \norm{q - s}^2 + \norm{s - r}^2 \\
    &= \norm{q - r}^2,
\end{align*}
which gives us that $\dist(r, p) < \dist(r, q)$. The other direction is symmetrical.

%\todo{Formalize} Proof sketch: We want to project $r$ onto the orange line. As long as $r \in H_p$ then the squiggly pink segment is shorter than the orange segment, which will make the green segment shorter than the red segment (which is what we want to show).
\end{proof}

\begin{cor} \label{prop:cellsareintersectionsofhalfplanes}
For every Voronoi cell we have
\[
    \mathcal{V}(p_i) = \bigcap_{\substack{1 \leq j \leq n \\ j \ne i}} h(p_i, p_j).
\]
\end{cor}
\begin{proof}
``$\subset$'': Let $r \in \mathcal{V}(p_i)$. Then $\dist(r, p_i) < \dist(r, p_j)$ for all $i \ne j$. Prop \ref{prop:hyperplaneinclusionproperty} then gives us that this is equivalent to $r \in h(p_i, p_j)$ for all $i \ne j$.

``$\supset$'': This argument is symmetrical to the above argument.
\end{proof}
A Voronoi cell is thus the intersection of convex sets and is therefore convex. We conclude that the Voronoi cells are open and convex (possibly unbounded) polygons with at most $n - 1$ vertices and $n - 1$ edges. \\

\section{Shape of the entire diagram}
We now look at the shape of the entire Voronoi diagram. From Corollary \ref{prop:cellsareintersectionsofhalfplanes} it follows that the edges of $\VorG(P)$ are made up of parts of straight lines, namely the bisectors between different points of $P$. We now classify these based on the structure of the points in $P$:
\begin{thm} \label{prop:structureofentirevoronoidiagram}
If the points in $P$ are collinear then $\VorG(P)$ consists of $n - 1$ parallel lines. Otherwise, $\VorG(P)$ is connected and its edges are either segments or half-lines.
\end{thm}
\begin{proof}
Assume that the points in $P$ are collinear. By applying an isometry to $P$, we may assume without loss of generality that the points of $P$ lie on the $x$-axis:
\[
    P = \curly{(x_1, 0), (x_2, 0), \ldots, (x_n, 0)},
\]
where we assume that $x_1 < x_2 < \cdots < x_n$ by rearranging the points if necessary. By definition, we have that $p \in \VorG(P)$ if and only if $p \not\in \mathcal{V}(x_i, 0)$ for all $i$. Let $(x, y) \in \R^2$ such that $x_i < x < x_{i+1}$. Then $(x, y) \in \VorG(P)$ if
\[
    \dist((x, y), (x_i, 0)) = \dist((x, y), (x_{i+1}, 0)).
\]
If furthermore $(x, y) \in \VorG(P)$ then we get
\begin{align*}
    &\norm{(x, y) - (x_i, 0)} = \norm{(x, y) - (x_{i+1}, 0)} \\
    \iff &\sqrt{(x - x_i)^2 + y^2} = \sqrt{(x - x_{i+1})^2 + y^2} \\
    \iff &\abs{x - x_i} = \abs{x - x_{i+1}}.
\end{align*}
Thus if $(x, 0) \in \VorG(P)$ then $(x, y) \in \VorG(P)$ for all $y \in \R$. This shows that $\bi((x_i, 0), (x_{i+1}, 0)) \subset \VorG(P)$ for all $i < n$. Every point of $\VorG(P)$ is on one of these bisectors, and the bisectors are all parallel, which proves the claim.

Assume that the points in $P$ are not collinear. First, we show that the edges of $\VorG(P)$ are either segments or half-lines. Suppose for a contradiction that there is an edge $e$ of $\VorG(P)$ that is a full line and assume that $e \subset \partial\mathcal{V}(p_i) \cap \partial\mathcal{V}(p_j)$. Let $p_k \in P$ be a point which is not collinear with $p_i$ and $p_j$. Then the line $\bi(p_j, p_k)$ is not parallel to the line $e$, hence they have an intersection point. Then there exists a point $v \in e \cap {}^{\circ}h(p_k, p_j)$. The situation is visualized here:
\[
    \includegraphics[scale=0.8]{diagram_is_connected} %\includegraphics[scale=0.22]{temp-fig-4}
\]
We have that $v \in \partial\mathcal{V}(p_j)$ by definition of $e$. Now note that
\[
    \partial \mathcal{V}(p_j) = \partial \para{\bigcap_{a \ne j} h(p_j, p_a)} \subset^{\footnotemark} \bigcup_{a \ne j} \partial h(p_j, p_a) = \bigcup_{a \ne j} \bi(p_j, p_a).
\]
\footnotetext{Here we used that $\partial(A \cap B) \subset \partial A \cup \partial B$.}As $v \in h(p_k, p_j)$ we have that $\dist(v, p_k) < \dist(v, p_j)$, hence $v \not\in \bi(p_j, p_k)$, so $v \not\in \partial{V}(v_j)$ by the above characterization of $\partial \mathcal{V}(p_j)$.
This is a contradiction, so $e$ can't be a full line. Now we show that $\VorG(P)$ is connected. Assume for the sake of a contradiction that $\VorG(P)$ is not connected. Then there exists a $\partial \mathcal{V}(p_i)$ which is not path connected. This can only happen if $\partial \mathcal{V}(p_i)$ consists of two parallel lines. This contradicts the fact that $\VorG(P)$ contains no lines. Thus $\VorG(P)$ is connected.
\end{proof}

Finally, we show that that the complexity of the vertices and edges is $\mathcal{O}(n)$:
\begin{thm} \label{thm:numberofvertsandedges}
For $n \geq 3$, the number of vertices in $\VorG(P)$ is at most $2n - 5$ and the number of edges is at most $3n - 6$.
\end{thm}
\begin{proof}
If the points in $P$ are collinear, then Theorem \ref{prop:structureofentirevoronoidiagram} implies the claim. Now assume that the points in $P$ are not collinear. As a first preprocessing step, we start by transforming $\VorG(P)$ into an actual plane graph, as some of the edges in $\VorG(P)$ may be half-lines. Let $v_1, \ldots, v_k$ denote the vertices of $\VorG(P)$. Let $p = \tfrac{1}{k}(v_1 + v_2 + \cdots + v_k) \in \R^2$ and let
\[
    r = 1 + \max\curly{\dist(p, v_1), \dist(p, v_2), \ldots, \dist(p, v_k)}.
\]
Then let $B_r(p) \subset \R^2$ denote the open ball with center $p$ and radius $r$. We have that $B_r(p)$ contains every vertex $v_i$ and that every half-line edge $e$ of $\VorG(P)$ intersects $\partial B_r(p)$ exactly once. Now define $v_{\infty} \in \R^2$ as any point in $\R^2 - B_r(p)$ and transform every half-line edge $e$ into a path with finite length by connecting the half-lines to the point $v_{\infty}$. This is possible since $\R^2 - \overline{B_p(r)}$ only contains these half-lines, and every half-line is pointing in a unique direction so we may then transform the half-lines in order by starting with those which are closest to $v_{\infty}$. An example of this construction is given here:
\[
    \includegraphics[width=\textwidth]{projective_embedding} % \includegraphics[scale=0.2]{temp-fig-5}
\]
In this way we can turn $\VorG(P)$ into a planar graph. For a planar graph $G$, Euler's polyhedra formula from topology states that
\begin{equation} \label{eq:eulerformulainproof}
    V - E + F = 2,
\end{equation}
where $V$ is the number of vertices, $E$ is the number of edges and $F$ is the number of faces of $G$. Let $n_v$ denote the number of vertices of the original $\VorG(P)$, and let $n_e$ denote the number of edges. In our modification, we only added a single vertex, so by plugging into (\ref{eq:eulerformulainproof}) we obtain the following relationship:
\begin{equation} \label{eq:eulersformulaapplied}
    (n_v + 1) - n_e + n = 2.
\end{equation}
Note that $n$ is the number of faces, since we have a Voronoi cell for each point in $P$. Every vertex $v$ in $G$ has $\deg(v) \geq 3$, otherwise there would be a $\mathcal{V}(p_i)$ which is not convex. This means that
\[
    \sum_{v \in V(G)} \deg(v) \geq 3 \abs{V(G)} = 3(n_v + 1).
\]
Now we want to compute the left side of the above inequality. Given a vertex $v$ we have that $\deg(v)$ counts the number of edges which touch $v$, and in $G$ every edge touches exactly 2 vertices, which gives us that $\sum_{v \in V(G)} \deg(v) = 2 n_e$. Combining these facts, we obtain the inequality:
\begin{equation} \label{ineq:eulersformulacompanion}
    2 n_e \geq 3(n_v + 1).
\end{equation}
Multiplying (\ref{eq:eulersformulaapplied}) by $2$, isolating $2 n_e$ and then applying (\ref{ineq:eulersformulacompanion}) we get:
\begin{align*}
    2 (n_v + 1) - 2 n_e + 2 n = 4
    &\iff 2 n_e = (2 n_v + 1) + 2n - 4 \\
    &\implies 3(n_v + 1) \leq 2 (n_v + 1) + 2n - 4 \\
    &\implies n_v \leq 2n - 5.
\end{align*}
Multiplying (\ref{eq:eulersformulaapplied}) by $3$, isolating $3 (n_v + 1)$ and then applying (\ref{ineq:eulersformulacompanion}) we get:
\begin{align*}
    3 (n_v + 1) - 3 n_e + 3 n = 6
    &\iff 3 (n_v + 1) = 3 n_e - 3n + 6 \\
    &\implies 2 n_e \geq 3n_e - 3n + 6 \\
    &\implies n_e \leq 3n - 6.
\end{align*}
This proves the theorem.
\end{proof}

\section{Characterizing bisectors in the diagram}
We have seen that we have a linear number of vertices and edges $\VorG(P)$, but we have a quadratic number of bisectors $\bi(p_i, p_j)$ of which every edge of $\VorG(P)$ is a subset of, and every vertex in $\VorG(P)$ is an intersection point of two such bisectors. Thus it would be interesting to characterize when a particular bisector is a part of $\VorG(P)$. First, we need a definition:

\begin{defn}[Largest empty circle]
For a $q \in \R^2$ we define $C_P(q)$ to be \emph{the largest empty circle of $q$ with respect to $P$}, which is the largest empty circle with $q$ as its center that does not contain any point of $P$ in its interior. Formally,
\[
    C_P(q) = B_r(q), \quad \text{where} \quad r = \sup\makeset{\lambda \in \R^+}{B_{\lambda}(q) \cap P = \varnothing}.
\]
\end{defn}
\[
    \includegraphics[scale=0.8]{largest_empty_circle}
\]
\begin{thm} \label{thm:characterizationofbisectors} The bisectors and their intersections are characterized by:
\begin{enumerate}[{(}i{)}]
    \item $q \in \R^2$ is a vertex of $\VorG(P)$ if and only if \[ \abs{\partial C_P(q) \cap P} \geq 3. \]
    \item $\bi(p_i, p_j)$ defines an edge of $\VorG(P)$ if and only if \[ \exists q \in \bi(p_i, p_j) \colon \partial C_P(q) \cap P = \curly{p_i, p_j}. \]
\end{enumerate}
\end{thm}
\[
    \includegraphics[scale=0.8]{vert_edge_char}
\]
\begin{proof}
We prove each statement individually:
\begin{enumerate}[{(}i{):}]
    \item ``$\Leftarrow$'': Let $q \in \R^2$ and assume that $\abs{\partial C_P(q) \cap P} \geq 3$. Let $p_i, p_j, p_k$ be three distinct points from $\partial C_P(q) \cap P$. Since $C_P(q) \cap P = \varnothing$ by definition, this means that $q$ is equally close to $p_i, p_j, p_k$ but not closer to any other points in $P$, so $q \in \partial\mathcal{V}(p_i) \cap \partial\mathcal{V}(p_j) \cap \partial\mathcal{V}(p_k) \subset \VorG(P)$, and it is a vertex since it is at an intersection of 3 or more bisectors.

    ``$\Rightarrow$'': Let $q \in \R^2$ be a vertex of $\VorG(P)$. A vertex of $\VorG(P)$ touches at least 3 different edges, and thus touches at least 3 distinct Voronoi cells $\mathcal{V}(p_i), \mathcal{V}(p_j)$ and $\mathcal{V}(p_k)$. So $q \in \partial\mathcal{V}(p_i) \cap \partial\mathcal{V}(p_j) \cap \partial\mathcal{V}(p_k)$. This gives us that
    \[
        \dist(q, p_i) = \dist(q, p_j) = \dist(q, p_k).
    \]
    Denote the above distance by $D$. Now assume for the sake of a contradiction that there exists $p_\alpha \in P$ such that $\dist(q, p_{\alpha}) < D$. Then there are parts of the bisectors $\bi(p_{\alpha}, p_i), \bi(p_{\alpha}, p_j), \bi(p_{\alpha}, p_k)$ contained inside $B_D(q)$, which means that $\mathcal{V}(p_i), \mathcal{V}(p_j), \mathcal{V}(p_k)$ do not all meet at $q$, a contradiction. This means that $C_P(q) \cap P = \varnothing$ and $p_i, p_j, p_k \in \partial C_P(q)$.

    \item ``$\Leftarrow$'': Let $q \in \bi(p_i, p_j)$ such that $\partial C_P(q) \cap P = \curly{p_i, p_j}$. So $C_P(q) \cap P = \varnothing$, which by definition of $C_P(q)$ means that
    \[
        \dist(q, p_i) = \dist(q, p_j) \leq \dist(q, p_k)
    \]
    for all $k$. So $q \in \VorG(P)$ and is either a vertex or an edge. Since $\abs{\partial C_P(q) \cap P} < 3$ part (i) gives us that $q$ is not a vertex, hence it must be an edge, which is a subset of $\bi(p_i, p_j)$. 

    ``$\Rightarrow$'': Let $e \subset \bi(p_i, p_j)$ be an edge of $\VorG(P)$. For $q \in e$ we have that $\dist(q, p_i) = \dist(q, p_j)$, and that $q$ touches $\mathcal{V}(p_i)$ and $\mathcal{V}(p_j)$. By applying the same contradiction proof as in (i) ``$\Rightarrow$'' we have that there is no point in $P$ which is closer to $q$ than $p_i$ and $p_j$, thus $\partial C_P(q) \cap P = \curly{p_i, p_j}$.
\end{enumerate}
\end{proof}

% % % % % % % % % % % % % % % % % % % % % % % % % % % % % % % % % % % %
%
% How to store a Voronoi diagram? DCELs
%
% % % % % % % % % % % % % % % % % % % % % % % % % % % % % % % % % % % % 

\section{The DCEL data structure}
The basic idea in this section is based on Section 2.2 in \cite{CompGeo}.

We want to write an algorithm to compute the Voronoi diagram, which leads us to a natural question: how do we store Voronoi diagrams on a computer?  We'll need the following geometric data structure:
\begin{defn}[DCEL] \label{defn:dcel}
A \emph{double connected edge list} (DCEL) is a data structure which represents a subdivision of $\R^2$. A DCEL consists of a lists of vertices, faces and edges. For every edge we will have two copies of it, with opposite orientations, so we will refer to each copy as a directed edge and call it a half-edge, so we actually store a list of half-edges. These three structures are represented as follows:
\begin{description}
  \item[\textsf{Vertex}] $v$ -- represents a vertex of the subdivision. Properties:
  \begin{itemize}
    \item $v.\textsf{position} \in \R^2$: Describes the position of $v$.
    \item $v.\textsf{edge} \text{ is a } \textsf{HalfEdge}$: Points to a half-edge which has $v$ as its start vertex.
  \end{itemize}
  \item[\textsf{Face}] $f$ -- represents a face of the subdivision. Properties:
  \begin{itemize}
    \item $f.\textsf{edge} \text{ is a } \textsf{HalfEdge}$: Points to a half-edge which lies on $\partial f$, and which is a part of a cycle of half-edges which goes around $f$ in counterclockwise order.
  \end{itemize}
  \item[\textsf{HalfEdge}] $e$ -- represents a half-edge of the subdivision. Properties:
  \begin{itemize}
    \item $e.\textsf{origin} \text{ is a } \textsf{Vertex}$: Since the half-edge is directed, we have a first and a second vertex in relation to the edge's direction, and this points to the first vertex.
    \item $e.\textsf{twin} \text{ is a } \textsf{HalfEdge}$: Points to the half-edge with the same vertices as $e$, but pointing in the opposite direction.
    \item $e.\textsf{face} \text{ is a } \textsf{Face}$: Points to the face which lies to the left of $e$.
    \item $e.\textsf{next} \text{ is a } \textsf{HalfEdge}$: Around $e.\textsf{face}$ we have a cycle half-edges which is oriented counterclockwise, and given $e$ in this cycle, $e.\textsf{next}$ gives us the next edge.
    \item $e.\textsf{prev} \text{ is a } \textsf{HalfEdge}$: Around $e.\textsf{face}$ we have a cycle half-edges which is oriented counterclockwise, and given $e$ in this cycle, $e.\textsf{prev}$ gives us the previous edge.
  \end{itemize}
\end{description}
\end{defn}

\begin{rmk}
In the CompGeo book the DCEL structure allows a face to have holes, but since Voronoi diagrams and Delaunay triangulations don't have holes in their faces, we have chosen to omit this feature.
\end{rmk}

\begin{ex}
Consider a graph $G$ with 9 vertices and 10 edges embedded into $\R^2$, which is given as the black figure in the following:
\[
    \includegraphics[width=\textwidth]{dcel_example} % \includegraphics[scale=0.37]{temp-fig-10}
\]
Then this induces a subdivision of $\R^2$ which we represent as a DCEL. The half-edges are given as the blue arrows, the faces as $f_1, f_2, f_3$ and the vertices are the vertices of $G$. Some of the pointers are visible on the figure.
\end{ex}

\begin{rmk} \label{rmk:boxalsohaslinearnumedges}
Note that the DCEL does not support infinite edges, so what we do is put a bounding box $B$ with some padding around the vertices of $\Vor(P)$, and then intersect the infinite edges and faces with the boundary of $B$ and only keep the part inside the bounding box.
\[
    \includegraphics[width=\textwidth]{voronoi_bounding_box} % \includegraphics[scale=0.25]{temp-fig-3}
\]
The aim of our algorithms will then be to calculate the DCEL in the right figure.

How does intersecting the edges of $\Vor(P)$ with such a bounding box $B$ affect the number of edges?
\[
    \includegraphics[scale=0.7]{dcel_edges_example}
\]
In the worst case, as depicted on the left figure, 4 new edges may be added to a single unbounded face. In the general case however, as depicted on the right figure, we only introduce between 1-3 edges per face. Thus Theorem \ref{thm:numberofvertsandedges} implies that the complexity is still linear.
%If we intersect $\VorG(P)$ with the bounding box $B$ then $B$ adds at most 2 edges to every cell: The bounded cells inside $B$ stay intact, and for unbounded cells we have two cases. Either a vertex of $B$ is contained in the cell, and then 2 edges will be added, otherwise a single edge is added. Hence in the worst case we add an edge for each point in $P$, and then an edge for each of the 4 corners of $B$, so the DCEL $\Delta$ representing $\Vor(P)$ then has at most
%\[
%    (3n - 6) + (n + 4) = 4n - 2 = \mathcal{O}(n)
%\]
%edges.
\end{rmk}
\chapter{Mathematical setup for Fortune's algorithm}

sdkfj
\chapter{Data structures for Fortune's algorithm}

lsdkf
\chapter{Description of Fortune's algorithm}

We are now ready to describe Fortune's algorithm. We start with describing an overview of the algorithm, and then in the next section we describe some of the details thoroughly -- so anytime the algorithm says ``see detail $n$'', then this detail can be found in the next section.

\begin{alg} \label{alg:fortune} \textsc{VoronoiDiagram}(P) \\
\textit{Input:} A set $P = \curly{p_1, \ldots, p_n}$ of point sites in the plane. \\
\textit{Output:} The Voronoi diagram $\Vor(P)$ given inside a bounding box in a doubly-connected edge list $\mathcal{D}$.
\begin{enumerate}
    \item Initialize the event queue $\mathcal{Q}$ with a site event for every point in $P$, initialize the beach line tree $\mathcal{T}$ to be \textsc{nil}, and let the DCEL $\mathcal{D}$ be empty.
    \item Repeat the following until $\mathcal{Q}$ is empty:
    \begin{enumerate}[i.]
        \item Remove the event $e$ with the largest $y$-coordinate from $\mathcal{Q}$.
        \item If $e$ is a site event call \textsc{HandleSiteEvent}$(e)$.
        \item If $e$ is a site event call \textsc{HandleCircleEvent}$(e)$.
    \end{enumerate}
    \item At this point the internal nodes in $\mathcal{T}$ represent the infinite edges of $\Vor(P)$. Compute a bounding box $B$ which contains all points in $P$, as well as all the vertices of $\Vor(P)$, which are contained in $\mathcal{D}$. Intersect the infinite edges in $\mathcal{T}$ with $B$ and let these intersection points be new vertices in $\mathcal{D}$. Add new edges and pointers to make sure we still have a proper DCEL structure.
\end{enumerate}
\end{alg}

\begin{proc} \textsc{HandleSiteEvent}$(e)$
\begin{enumerate}
    \item Let $p_i$ denote the site that $e$ points to.
    \item If $\mathcal{T} = \textsc{nil}$ then let $\mathcal{T}$ store the single arc that is described by $p_i$ and return.
    \item Otherwise, $\mathcal{T} \ne \textsc{nil}$. Search in $\mathcal{T}$ for the arc $\alpha$ vertically above $p_i$, that is the arc at which the vertical line through $p_i$ intersects the beach line. (See detail \#)
    \item If $\alpha$ has a pointer to a circle event $e'$, then remove $e'$ from $\mathcal{Q}$, as this circle event is now a false alarm since $\alpha$ is about to disappear earlier than we initially thought.
    \item Create the new arc $\beta$ defined by $p_i$ and insert it into $\mathcal{T}$ as described in Section \ref{sec:insertingatsiteevents}. Update $\mathcal{D}$ by creating the new half-edges which will be traced out by the two new breakpoints as described in Section \ref{sec:dcelatsiteevents}.
    \item Check the triple of consecutive arcs where the new arc for $p_i$ is the left arc to see if the breakpoints converge. If so, insert the circle event into $\mathcal{Q}$ and add pointers between the node in $\mathcal{T}$ and the node in $\mathcal{Q}$. Do the same for the triple where the new arc is the right arc. (See detail \#)
\end{enumerate}
\end{proc}

\begin{proc} \textsc{HandleCircleEvent}$(e)$
\begin{enumerate}
    \item Let $\alpha$ be the arc pointed to by $e$, which is about to disappear from the beach line.
    \item Delete all circle events from $\mathcal{Q}$ which involve $\alpha$: The one where $\alpha$ is the middle arc has already been deleted, and the other two possible circle events where $\alpha$ is the left and right arc respectively can be found through $\alpha$'s \textsf{.leftArc} and \textsf{.rightArc} pointers. (See detail \#)
    \item Delete $\alpha$ from $\mathcal{T}$, how this is done is described in Section \ref{sec:deletingatsiteevents}.
    \item Add the center $c$ of the circle describing $e$ as a new vertex of $\mathcal{D}$. Connect the half-edges in $\mathcal{D}$ that converge at $e$, and create a new half-edge which starts at $c$ and setup the appropriate pointers. The details are given in Section \ref{sec:dcelatcircleevents}.
    \item As $\alpha$ disappears from the beach line, we get new triples of consecutive arcs which might have converging breakpoints that can lead to a circle event. Check these and add circle events if needed. (See detail \#)
\end{enumerate}
\end{proc}

\section{Details}

\subsection*{Detail 1: Intersecting lines}
As a subroutine in several steps during the algorithm we will need to intersect line segments or rays with each other. We start by describing a solution to this in general. This detail assumes that the reader is familiar with basic linear algebra. We want to find the intersection between 2 lines. We parametrize the lines as follows:
\[
    \gamma_1(t) = p + t d_1 \quad \text{and} \quad
    \gamma_2(s) = q + s d_2,
\]
where $p$ and $q$ are points on the lines, and $d_1$ and $d_2$ are direction vectors which tells us which way the lines point. The situation is illustrated as follows:
\[
    \includegraphics[scale=0.8]{intersect_lines}
\]
To find an intersection point, we must find $s, t \in \R$ such that
\[
    \gamma_1(t) = \gamma_2(s).
\]
That is, we want to solve
\[
    p + t d_1 = q + s d_2.
\]
This can be rewritten into the matrix equation
\[
    A \begin{pmatrix}
        s \\
        -t
    \end{pmatrix}
    =
    q - p,
\]
where $A = \begin{pmatrix} \mid & \mid \\ d_1 & d_2 \\ \mid & \mid \end{pmatrix}$ is the $2 \times 2$ matrix which has $d_1$ and $d_2$ as left and right columns, respectively. The equation system has a unique solution if $d_1$ and $d_2$ are linearly independent, and if they are, the solution is given by
\[
    \begin{pmatrix}
        s \\
        -t
    \end{pmatrix}
    =
    A^{-1} (q - p).
\]
This linear independence property is equivalent to checking that the determinant $\det(A)$ is non-zero. So, in order to check if two lines intersect, we first check if $\det(A) \ne 0$. If not, we say the lines don't intersect. Otherwise, they intersect, and we use the above solution to find the intersection point.

Now in the case of line segments and rays, we also need some constraints on $s$ and $t$, it is not enough that the lines themselves intersect.

To intersect line segments, where the first line segment is given as the points between $p_1$ and $p_2$, and the second line segment is given as the points between $q_1$ and $q_2$, then we let $p = p_1$ and $q = q_1$ and then set $d_1 = p_2 - p_1$ and $d_2 = q_2 - q_1$ and then we solve the equation system as above. If there is a solution to the equation system, then we also need to check that $s \in [0, 1]$ and $t \in [0, 1]$ in order for the intersection point to lie on both of the line segments.

Finally, we look at intersecting two rays. We describe the rays as the lines at the start of this detail, but now we have the requirement that $s, t \geq 0$, which gives us that the rays start at $p$ and $q$, and then they shoot out in the direction of $d_1$ and $d_2$, respectively.

\subsection*{Detail 2: Choose breakpoint based on the ordering of tuple}
Let $p_i$ and $p_j$ be two sites and let $\beta_i$ and $\beta_j$ be the hyperbolas that they describe. If a breakpoint stores the tuple $(p_i, p_j)$ then we want a way to find the $x$ coordinate of that breakpoint. Since the intersection of two hyperbolas may contain 2 intersection points, we need to pick the correct one. We already described this when discussing internal tree nodes at the start of Section \ref{sec:bst}, but let's recap: The order is important since the intersection of the hyperbolas defined by $p_i$ and $p_j$ consists of two points, and the order lets us tell these breakpoints apart. If we consider the beach line as running from the left to the right, then at every breakpoint an arc is leaving, and another is entering it. Thus the tuple $(p_i, p_j)$ tells us that we are interested in the breakpoint at which an arc pointing to $p_i$ leaves, and an arc pointing to $p_j$ is entering the beach line. We will need the following result:

\begin{prop} \label{prop:highschool1}
Let $f(x) = a x^2 + b x + c$ be a polynomial with discriminant $D > 0$ with roots $r_1 < r_2$. Then $r = \tfrac{1}{2}(r_1 + r_2)$ is the only solution to $\displaystyle\frac{df}{dx}(r) = 0$ and the expressions $\displaystyle\frac{df}{dx}(r_1)$ and $\displaystyle\frac{df}{dx}(r_2)$ are non-zero and have opposite signs.
\end{prop}
This fact can be visualized as follows:
\[
    \includegraphics[scale=0.8]{polynomial_result}
\]
\begin{proof}[Proof of Proposition \ref{prop:highschool1}]
It is well-known that we may factor $f$ as follows:
\[
    f = a (x - r_1) (x - r_2) = a x^2 - a(r_1 + r_2) x + a r_1 r_2.
\]
Since two polynomials are equal if and only if their coefficients are equal we get $b = - a (r_1 + r_2)$, which gives us
\[
    \frac{df}{dx} (r) = 2 a r + b = 2 a \para{\frac{r_1 + r_2}{2}} - a (r_1 + r_2) = 0.
\]
This is the only solution since $\displaystyle\frac{df}{dx}$ is a first degree polynomial. Now note that $\displaystyle\frac{d^2 f}{d x^2}(x) = 2a \ne 0$ and $r_1 < r < r_2$ which gives us that
\[
    \text{sgn} \para{\frac{df}{dx}(r_1)} = -\text{sgn} \para{\frac{df}{dx}(r_2)} \ne 0.
\]
\end{proof}
When intersecting two of the paraboals of the beach line, we will find two intersection points, because of our assumptions. Proposition \ref{prop:highschool1} then gives us that at these intersection points $r_1, r_2$ we have that
\[
    \begin{cases}
        \displaystyle\frac{d(\beta_i - \beta_j)}{dx}(r_k) \ne 0 \text{ for } k = 1, 2 & \text{ }\vspace{0.25cm} \\ \text{sgn} \para{\displaystyle\frac{d(\beta_i - \beta_j)}{dx}(r_1)} = -\text{sgn} \para{\displaystyle\frac{d(\beta_i - \beta_j)}{dx}(r_2)} & \text{ }
    \end{cases}
\]
We then want to locate a specific breakpoint between two arcs, and the above will help us to do this.

To intersect the two parabolas $\beta_i$ and $\beta_j$ we write
\[
    (\beta_i - \beta_j)(x) = a x^2 + b x + c,
\]
where (for $p = p_i$, $q = p_j$, $h_p = p_y - \ell_y$ and $h_q = q_y - \ell_y$)
\begin{align*}
    a &= \frac{1}{2} \para{\frac{1}{h_p} - \frac{1}{h_q}}, \\
    b &= \frac{q_x}{h_q} - \frac{p_x}{h_p}, \\
    c &= \frac{q_y(p_x^2 + p_y^2) - p_y (q_x^2 + q_y^2) + \ell_y (q_x^2 + q_y^2 - p_x^2 - p_y^2) + \ell_y^2 (p_y - q_y)}{2 h_p h_q}.
\end{align*}
The square root of the discriminant is then
\[
    d = \sqrt{b^2 - 4 ac} = \sqrt{\frac{(p_x - q_x)^2 + (p_y - q_y)^2}{h_p h_q}}.
\]
The $x$-values of the intersection points are then given by the well-known formulas
\[
    r_1 = \frac{-b - d}{2 a}, \quad
    r_2 = \frac{-b + d}{2 a},
\]
which gives us the intersection points $q_1 = (r_1, \beta_i(r_1))$ and $q_2 = (r_2, \beta_i(r_2))$. Now, we want to find the breakpoint which at which an arc of $\beta_i$ exits the beach line, and an arc of $\beta_j$ enters the beach line. Proposition \ref{prop:highschool1} gives us a way of picking which one of $q_1$ and $q_2$ is the breakpoint that we need. For $\beta_i$ to exit and $\beta_j$ to enter, we need to pick $k$ such that
\[
    \frac{d \beta_i}{dx}(r_k) > \frac{d \beta_j}{dx}(r_k).
\]
This is illustrated in the following figure, with a slight abuse of notation:
\[
    \includegraphics[scale=0.8]{derivativedirections}
\]
Proposition \ref{prop:highschool1} guarantees that either
\begin{align*}
    \frac{d \beta_i}{dx}(r_1) > \frac{d \beta_j}{dx}(r_1) &\text{ and } \frac{d \beta_i}{dx}(r_2) < \frac{d \beta_j}{dx}(r_2) \\
    &\text{or} \\
    \frac{d \beta_i}{dx}(r_1) < \frac{d \beta_j}{dx}(r_1) &\text{ and } \frac{d \beta_i}{dx}(r_2) > \frac{d \beta_j}{dx}(r_2),
\end{align*}
so it is possible to make the right choice. Now, note that by some simple algebraic manipulations we have that
\[
    \frac{d \beta_i}{dx}(r_k) > \frac{d \beta_j}{dx}(r_k)
\]
if and only if
\[
    (r_k - p_x) (q_y - \ell_y) > (r_k - q_x) (p_y - \ell_y).
\]
This gives us a criterion for deciding which intersection point describes the breakpoint in question, and this is the criterion used in the implementation.

\subsection*{Detail 3: How to find the arc vertically above a point}
At a site event when we discover a new point $p$ we want to find the arc $\alpha$ vertically above $p$, as illustrated here:
\[
    \includegraphics[scale=0.8]{locate_arc}
\]
Let $x_1, x_2, \ldots, x_k$ denote the breakpoints on the beach line. These are stored as internal nodes in our BST $\mathcal{T}$. Since the keys for the internal nodes are the $x$-values of the breakpoints, we may locate the arc $\alpha$ using binary search in $\mathcal{T}$. Starting at an internal node $x$ in $\mathcal{T}$ we visit its left subtree if $x\textsf{.key} < p_x$, and we visit its right subtree if $x\textsf{.key} \geq p_x$. The key property is computed at every check, since it is a function of the current position of the sweep line, see Detail 2 for how the key is computed. Eventually we will reach the leaf which stores the arc $\alpha$.

\subsection*{Detail 4: How to check if two breakpoints are converging}
Let $p, q, r$ be three sites from $P$ which define 3 consecutive arcs on the beach line. Let $x$ and $y$ be two breakpoints, where $x$ is sliding along $\bi(p, q)$ and $y$ is sliding along $\bi(q, r)$ as we vary $\ell$. We want to check whether $x$ and $y$ converge, and if so, what is the location of their intersection, and when during the sweep of $\ell$ will this occur. The two possible scenarios are illustrated below:
\begin{figure}[H]
    \centering
    \subfloat{
      \includegraphics[scale=0.6]{converging_breakpoints_example}
    }
    \subfloat{
      \includegraphics[scale=0.6]{converging_breakpoints_nonexample}
    }
\end{figure}
In the divergent case we include the case where the bisectors are collinear. 

To check for convergence, we transform the problem into a problem of intersecting two rays. Let $x$ and $y$ denote the current location of the breakpoints, and let $x'$ and $y'$ denote the breakpoints new positions after moving the sweep line some arbitrary amount downwards, and then let $d_1 = x' - x$ and $d_2 = y' - y$. Then $s \mapsto x + s d_1$ and $t \mapsto y + t d_2$ parametrize $\bi(p, q)$ and $\bi(q, r)$, respectively. The setup, in the case where the rays do intersect, looks like this:
\[
    \includegraphics[scale=0.8]{ray_intersection}
\]
Now, as we saw in Detail 1, then the two rays converge if $t \geq 0$ and $s \geq 0$.

This can be interpreted as follows: If $s$ is positive, then that means that $x$ will hit $y$ in the future, and likewise if $t$ is positive, then $y$ will hit $x$ in the future. This is important as we are treating the events chronologically, and if $x$ and $y$ already intersected in the past (or the lines they describe, rather) then they can define no future circle event.

\subsection*{Detail 5: Finding a circle through 3 points}
As a part of the algorithm, we need to find the circle $C$ through 3 points $p, q, r$. It turns out if we intersect $\bi(p, q)$ and $\bi(q, r)$ we find the center of $C$, and then to find the radius we just need to find the distance from the center to one of the points. This is because if $x \in \bi(p, q) \cap \bi(q, r)$ then
\[
    \dist(x, p) = \dist(x, q) = \dist(x, r),
\]
so $x$ is exactly the center of a circle through $p, q, r$. This is illustrated in this figure:
\[
    \includegraphics[scale=0.8]{circle_3_points}
\]
To intersect the bisectors, we form the midpoints
\[
    m_1 = \frac{1}{2}(p + q) \quad \text{and} \quad m_2 = \frac{1}{2}(q + r)
\]
and then we let $d_1$ and $d_2$ denote $q - p$ and $r - q$ rotated 90 degrees counterclockwise. Then $s \mapsto m_1 + s d_1$ and $t \mapsto m_2 + t d_2$ parametrize $\bi(p, q)$ and $\bi(q, r)$, respectively. Then we solve the linear system as in Detail 1.

\newpage
\subsection*{Detail 6: Deleting false alarms during a circle event}
At a circle event an arc disappears from the beach line, along with two breakpoints. Consider the following example, where at one time we have the arcs $\alpha_1, \alpha_2, \ldots, \alpha_7$ on the beach line along with the breakpoints $x$ and $y$ that $\alpha_5$ lies inbetween, and then after a circle event the arc $\alpha_5$ disappears after the breakpoints $x$ and $y$ intersect and get replaced by a new breakpoint $z$:
\[
    \includegraphics[scale=0.8]{circle_event_beachline_merge}
\]
When this happens, we have to remove the circle events that involve the breakpoints $x$ and $y$ merging with any other breakpoints.

\subsection*{Detail 7: Intersecting a bounding box with the in-progress DCEL to get the final DCEL}

\section{Correctness}
\begin{lem}
Algorithm \label{alg:fortune} can be implemented such that it runs in $\mathcal{O}(n \log n)$ time and uses $\mathcal{O}(n)$ storage.
\end{lem}
\begin{proof}
\todo{.}
\end{proof}
\chapter{Demo}

A demo of our JavaScript implementation of Fortune's algorithm is available at \url{http://funbyjohn.com/voronoi/}. In general you can add points by clicking with the mouse, and if you click on a point while holding down shift you delete the point.
\[
    \includegraphics[width=\textwidth]{demo1}
\]
The first demo can be seen above. Here the Voronoi diagram is enclosed in a bounding box, and you may hide the bounding box by pressing the 6 key, and you can change its size by pressing the 7 and 8 keys.
\[
    \includegraphics[width=\textwidth]{demo2}
\]
The second demo can be seen above. It shows the beach line, and you can move it up and down using the mouse. If you hold down shift the sweep line will move slower.
\[
    \includegraphics[width=\textwidth]{demo3}
\]
The final demo showcases the Delaunay triangulation, which is computed by taking the dual of the Voronoi graph.

\subsection*{Key legend}
Here is an overview displaying what every key does:
\begin{table}[H]
\centering
\begin{tabular}{ll}
\textbf{Key} & \textbf{Effect}                                           \\ \hline
1            & Toggle drawing the binary search tree in Demo 2           \\
2            & Toggle showing names of breakpoints in Demo 2             \\
3            & Toggle showing the oriented edges in the DCEL in Demo 2   \\
4            & Toggle showing face pointers in the DCEL in Demo 2        \\
5            & Toggle showing .next/.prev pointers in the DCEL in Demo 2 \\
6            & Toggle bounding box display in both Demo 1                \\
7            & Make bounding box smaller in Demo 1                       \\
8            & Make bounding box bigger in Demo 1                        \\
9            & Toggle showing circle events in Demo 2                    \\
Space        & Lock/unlock sweep line in Demo 2                          \\
Shift        & Hold down to make sweep line move slower in Demo 2        \\ \hline
\end{tabular}
\end{table}
Note that the face pointers and the .next/.prev pointers needs the oriented edges to be visible in order to work.

\subsection*{Warnings}
There are some bugs and numerical issues in the implementation at the time of writing. For a good time make sure to keep the following rules:
\begin{itemize}
    \item Make sure no points have the same $y$-value.
    \item Make sure that at most 3 points lie on the same circle.
    \item Don't place more than one point in the same spot.
\end{itemize}

\appendix
\chapter{Notation}

\begin{table}[H]
\begin{tabular}{ll}
$X - Y$ & Set difference \\
$\abs{X}$ & The number of elements in a finite set $X$. \\
$\iff$ & If and only if \\
$\implies$ & Implication \\
$\mathbb{R}$ & The real numbers. \\
$\mathbb{R}^n$ & The vector space of $n$-tuples of real numbers. \\
$\norm{\,\cdot\,}$ & Norm. \\
$\norm{\,\cdot\,}_p$ & The $L^p$ norm. \\
$\abs{x}$    & Absolute value if $x$ is a number. \\
$\dist(p, q)$ & The distance between $p$ and $q$, given by $\norm{p - q}$. \\
$\ip{\,\cdot\,}{\,\cdot\,}$ & An inner product. \\
$\subset$    & Subset (not strict, e.g. $A = B \implies A \subset B$). \\
$P$ & A set of points $\curly{p_1, p_2, \ldots, p_n}$ that we want to apply an algorithm to. \\
$p_i$ & A point in $P$ (see above). \\
$n$ & If not otherwise specified, $n$ is the number of points in $P$ (see above). \\
$\Vor(P)$    & The Voronoi diagram of $P$. \\
$\mathcal{V}(p_i)$    & The $i$th Voronoi cell. \\
$\VorG(P)$    & Refers to $\R^2 - \Vor(P)$. \\
$\mathcal{O}(f(n))$ & Big $O$-notation. \\
$\bi(p,q)$    & Bisector of $p$ and $q$. \\
$h(p,q)$    & Open half-plane containing $p$ with $\bi(p,q)$ as boundary. \\
$\overline{X}$ & The closure of a set $X \subset \R^n$, given by the union of $X$ with its limit points. \\
${}^{\circ}X$ & The interior of a set $X \subset \R^n$, given by the union of all interior points of $X$. \\
$\partial X$  & The boundary of a set $X \subset \R^n$, given by $\overline{X} - {}^{\circ}X$. \\
$\overline{B_r(p)}$ & $ = \makeset{x \in \R^n}{\dist(x, p) \leq r}$, the closed ball with center $p$ and radius $r$. \\
$B_r(p)$ & $ = \makeset{x \in \R^n}{\dist(x, p) < r}$, the open ball with center $p$ and radius $r$. \\
$\partial{B_r(p)}$ & $ = \makeset{x \in \R^n}{\dist(x, p) = r}$, the circle with center $p$ and radius $r$. \\
$V(G)$ & The set of vertices for the graph $G$. \\
$E(G)$ & The set of edges for the graph $G$. \\
$\deg(v)$ & The degree of a vertex $v$ in a graph, e.g. the number of edges that touch $v$.
\end{tabular}
\end{table}

\bibliographystyle{plain}
\bibliography{references}

\end{document}